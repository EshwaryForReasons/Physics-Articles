\documentclass[12pt, letterpaper, openany]{article}

\usepackage{parskip}

\title{Newton's Laws Of Motion}
\author{Eshwary Mishra}
\date{}

\setlength{\parindent}{0pt}

\begin{document}
\maketitle

Newton was certainly a brilliant man. Amongst his many achievements lays the creation of an enormously successful scheme upon which natural philosophy was subsequently based for 3 centuries proceeding. Combined with the equivalent schemes of Lagrange and Hamilton, which allow us to easier approach otherwise grotesque problems, we form \emph{Classical Mechanics}.

Classical mechanics was so successful and did a job so excellent at explaining the phenomena of everyday life, the mere thought of it being fundamentally incorrect did not strike natural philosophers even tangentially.

Of course, we live in a different era presently. Natural philosophers are now called Physicists and we have since found all the physics required for explaining the phenomena of daily life. The currently curiosity for physicists is quantum physics. Even after a hundred years of investigation, it is hardly understood and continues to puzzle the minds of the most brilliant researchers.

Today, however, I plan to focus only on the laws of Newton. While we now understand classical mechanics is not fundamental, it serves as an excellent approximation for the macroscopic world and is tremendously useful in explaining everyday phenomena.

Newton's Laws are the following:
\begin{enumerate}
\item An object will persist in its kinematic state in a straight line unless acted upon by an external force.

\item The sum of all forces upon an object

\item For every force, there exists another force equal in magnitude and opposite in direction acting upon the same object.
\end{enumerate}

The observant reader may notice I have provided the axioms in an uncommon form. I will now provide justification for this. The laws of Newton are often generalized and simplified to an extent where they no longer represent the true word of Newton. To demonstrate this, I will work through the laws backwards.

\section{The Third Law}

This law is unfortunately often stated in the following way: for every action there is an equal and opposite reaction. This is an unfortunate misrepresentation of Newton's original verbiage prone to myriad misconceptions.

First, the idea of action and reaction implies a time delay. For example, imagine someone you utterly hate. Now, perhaps sadistically, imagine throwing a large rock at them. Depending on your hatred of the person, you are likely now imagining either a rather gruesome scene or perhaps you are imagining that the person was able to dodge your attack. In either case, likely your hypothetical of this violent scene involved the person attempting to block or dodge your attack. This attempt of the victim we dub the \emph{reaction} to your \emph{action} of throwing the rock. If the victim were to then react and perhaps lunge at you, then we similarly label that a \emph{reaction}. It is clear that in modern English vernacular the terms action and reaction contain with them a temporal connotation. The reaction comes \emph{after} the action.

The third law makes no such claim. Quite the opposite, in fact. The law claims that the two forces exist \emph{simultaneously}. If one force grows bigger (say you push harder on an object) then the other force grows bigger simultaneously. Not after, not a little bit after, not directly after, not an infinitesimal time interval after, not an instant after, but simultaneously.

My Classical Mechanics professor made this a real point in our class. We were to use only the terminology of third-law pairs and companion forces. While this seemed unnecessary and tedious initially, I now appreciate it greatly. It helped greatly in avoidance of the unnecessary misconception on a time delay. And indeed, I do still believe that companion forces is the correct term for the pair of forces. Of course, to a physicist, this does not matter. They know exactly what the third law means and terminology is largely irrelevant.

\end{document}